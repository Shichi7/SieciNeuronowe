\documentclass[17pt]{article}
\usepackage[utf8]{inputenc}
\usepackage[T1]{fontenc}
\usepackage{multirow}
\usepackage{array}

\title{\textbf{SIECI NEURONOWE\\Sprawozdanie - Ćwiczenie 4}}
\author{Aleksander Poławski\\Grupa - Poniedziałek 18:15\\Prowadzący - mgr inż. Jan Jakubik}
\date{13 grudzień, 2020}

\usepackage[none]{hyphenat}%%%%
\setlength{\parindent}{0ex} 
\sloppy

\usepackage{caption}
\captionsetup[table]{name=Tabela}

\usepackage{geometry}
 \geometry{
 a4paper,
 total={170mm,257mm},
 left=25mm,
 top=25mm,
 right=25mm,
 bottom=25mm
 }

\begin{document}

\maketitle	

\section{Cel ćwiczenia}
Celem ćwiczenia czwartego laboratoriów kursu Sieci Neuronowe było poznanie podstawowych operacji realizowanych w sieciach konwolucyjnych, sposobu przetwarzania informacji wejściowej i sposobu uczenia takich sieci.

\section{Plan ćwiczenia oraz badań}

\begin{enumerate}
\item[a)] implementacja architektury sieci konwolucyjnej i fazy przesyłania w przód 

\item[b)] implementacja uczenia sieci konwolucyjnej

\item[c)] przeprowadzenie eksperymentów badających skuteczność implementowanych metod i porównanie wyników z sieciami implementowanymi w poprzednich zadaniach

\end{enumerate}

\section{Opis zaimplementowanego programu}

Do wykonania zadania rozwinięto program implementowany do zadania drugiego i trzeciego laboratorium.\\

Program zaimplementowano w środowisku PyCharm w języku Python, korzystając między innymi z bibliotek Numpy do przetwarzania obliczeń macierzowych.\\

Przed wykonaniem zadania w jego skład wchodziły następujące elementy:
\begin{itemize}
\item klasa Loader - umożliwia wczytywanie, przechowywanie i konwersję zbiorów uczących i testowych. Zawiera proste funkcje pomagające stwierdzić poprawność wczytania zbiorów.
\item klasa MLP - zawiera całą logikę tworzenia, ustawień, uczenia i testowania sieci MLP
\item klasa MLPLayer - zawiera całą logikę tworzenia, ustawień i działania poszczególnych warstw sieci
\item plik main - manager programu - organizujący kolejność wykonywania zadań programu, zawierający predefiniowane testy potrzebne do wykonania badań przewidzianych w ćwiczeniu
\end{itemize}

W trakcie wykonywania zadania dokonano następujących zmian:
\begin{itemize}
\item klasa Loader - rozbudowano klasę o możliwość wczytywania zbiorów testowych i treningowych w formacie odpowiednim na wejście warstwy konwolucyjnej
\item klasa Convo - dodano klasę zawierającą całą logikę tworzenia, ustawień, architekturę sieci konwolucyjnych
\item klasa ConvoLayer - dodano klasę zawierającą całą logikę tworzenia, ustawień i działania poszczególnych warstw sieci, w tym możliwość konwolucji, poolingu i spłaszczania map cech na wyjście do warstw w pełni połączonych
\end{itemize}

\newpage

\section{Podsumowanie}
\vspace{4mm}

Niestety zadanie zrealizowano tylko w zakresie implementacji architektury sieci konwolucyjnej oraz fazy przesyłania w przód. W związku z tym niemożliwe było wykonanie badań (bez implementacji uczenia) oraz porównanie wyników z siecią MLP implementowaną w poprzednich zadaniach.\\

W toku wykonywania ćwiczenia rozwinięto swoją wiedzę na temat elementarnych pojęć dotyczących sieci neuronowych poznając zasadę działania i architekturę sieci konwolucyjnej.


\end{document}